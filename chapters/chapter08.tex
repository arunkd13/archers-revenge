\chapter{}

\lettrine{T}{irupati} is a historic temple town, located in South India, in the
state of Andhra Pradesh. Though the temple of Lord Venkateswara, an incarnation
of Lord Vishnu, is located on the hills, the town itself is split into Tirupati
town, located below the hills and Tirumala town, over the hills. Since the town
is an important pilgrimage centre, millions of Hindu devotees visit the temple
every year. The queue for darshan (having a glimpse at the statue of the God) is
generally long and on an average a devotee spends many long hours waiting for
it. The town is swarmed by pilgrims throughout the year, and is well-connected
through air, trains and buses.

Tirumala is situated on the top of the hills which is reachable by cars and
buses, by road. There is another way to reach the top of the hills from the town
which thousands of people use regularly—by walking through the hills. Before
roads were made, this route was the only way to reach Tirumala. For an average
person, it takes 4 to 6 hours to climb up the hills by walk.

Many people still use the stairs to walk all the way to the top of the hill. The
walking route is covered with steps, cemented pathways and roof, from the bottom
of the hill to the top. Since people walk up and down during the nights too, the
route has been electrified. There are various shops located on the sides of the
stairs. If something is unavailable, shopkeepers shuttle between Tirupati town
and their hill-top shops to bring it, on request. On his first visit, Aryan
noticed all these things and was convinced that his supplies, groceries,
vegetables could easily be procured from the vendors on this route. He didn't
have to go to Tirupati town or Tirumala town for that. This was important for
him to evade the cops.

Not many people know that there were other walking routes that were used to
traverse the hill, during ancient times, but mostly abandoned now. One such
route interspersed with the present walking route. Aryan had identified it using
Google Maps, but he couldn't locate it initially. With the help of his
smartphone and GPS tracker, he identified the ancient route that was few hundred
meters away from the current route, through the trees.

Aryan immediately set off to explore it, which led him deeper into the jungle.
Initially he was intimidated by the expanse of nature, but as he moved deeper
into her lap, he felt like having reached home.

He was looking for two things—a place to live and a fresh water source. Local
people had informed him about a stream that flowed from the top of the mountain
and emptied into a waterfall below. With the help of Google Earth, he identified
that this stream intersected with the old unused route on which he was walking
right now. His goal was to reach that point.

After walking downhill for many hundred meters on the old unused route, he found
the stream. This was the fresh water source he was looking for. The stream
crossed his path and the wooden bridge used to walk over it was now broken. He
couldn't cross the bridge, but he didn't want to cross it anyway.

He had to find some place just before this stream where he could live. He was
looking for a cave or shelter and luckily, he found an old traveller's rest
house. This one-storey stone structure—possibly built to enable travellers on
foot to rest for sometime during the ancient times—offered a perfect place for
him to stay safely in this remote location, amidst the forest of the hills.

The rest house was built over a raised platform with roof and pillars. A thick
stone wall made of large grey stones covered almost all the four sides except
the entrance which was left open, without a door. There were two steps leading
into the house. There were open windows in the middle of all three side walls
for light and ventilation. Since the building was old and unused for a long
time, it was partially covered with vegetation, and he found a few bats resting
on the top. He cleaned the house and this structure was more than what he hoped
to find in this remote location. It was the perfect home he was searching for!

Since he had to go back before it was dark, he traced his way through the old
unused route and reached the new route where people were walking. After walking
downstairs for a couple of hours, he reached his hotel room and took a long nap.
He had a lot of planning and shopping to do over the next couple of days.