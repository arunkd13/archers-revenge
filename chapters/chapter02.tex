\chapter{}

\lettrine{I}{n India}, people usually become successful politicians because they are the kin
of successful ministers or MLAs, related to or friends of industrialists, or famous
film stars or social activists.

Guru was none of them.

He was the eldest among six siblings. When he was nineteen years old, his
father, a construction labourer, died in an accident. Guru was obliged to quit
his Government College education and join his mother in the construction work
to support his family.

Initially, he was content with the daily wages that were provided. Since his
brothers and sisters were still studying in a government school in Chennai,
they had access to free meals provided by the State. Money earned by Guru and
his mother was sufficient to feed the family in the mornings and at night.
But, the income of the family was not enough to arrange her sisters' marriages.

In India, parents need to provide dowry, usually in the form of gold jewellery
or cash, to get their
daughters married. The value of the dowry determines the status of the groom.
This practice was not prevalent in all communities, but in Guru's case, dowry
was very much a reality. Guru knew that he had to earn considerably more, and that
too in a short period, if he wanted to get his sisters married into
relatively well-to-do families.

Guru noticed that some workers occasionally died or suffered major health
ailments due to the poor safety standards maintained by construction companies at
their sites. The contractor he worked for did not bother about the safety of
workers—he did not consider investing in the well-being of labourers to be a good
investment. Guru also noticed that there was no labour union to fight for the
rights of labourers. He formed an informal labour union and started negotiating
with his employer on behalf of the affected families. He tried his best to secure
monetary compensation for the families of labourers who died or suffered serious
ailments due to their strenuous working conditions.

Initially, he faced a lot of opposition from his employer for his attempts to
bring justice to the labourers. He was attacked by thugs hired by his employer,
but he managed to survive the assault. The incident only increased his
determination to fight. His efforts were noticed by other labourers, and he formed
a network of able-bodied labourers to fight any such attacks in the
future. Unable to take him down by force, his employer relented and offered
monetary compensation to the affected families. This turned Guru into a hero among the
labourers, and he became their saviour. Labourers now consulted him about their problems
concerning company management, and he became their unelected representative.

However, Guru found that he was unable to make any money out of these endeavours.
What he did amounted to social service, not business. Since he had secured the
trust of labourers, Guru decided that it was time to capitalise on that trust. He now
negotiated with both parties to strike the best deal possible—for the
employer. He was paid a handsome commission for these cost-reduction efforts.
Guru found that he was not only able to make a decent amount of money, but was
also much sought after and thanked by both labourers and the employer. Even
though he didn't realise it, he was already learning the art of politics.

“Why should I ask my workers to vote for you? What have you done for our company
over the last few years as a ward councillor?” Guru heard his boss fuming over
the phone as he entered the general manager's cabin. The boss said, “Look at
this Guru—this useless ward councillor won the previous election because I
recommended his name to all our labourers. But, he conveniently forgot us
after that. Now he wants me to canvass for him again. What makes him think that
I will agree to it? I really don't understand. Anyway, what brings you here?”

Guru said, “You called me this morning to discuss the compensation to be
given to—”

His boss interrupted. “Oh, that. You do the best you can. I know you'll have
the company's interest as your topmost priority. Am I right?”

Guru said, “Sir, I have never forgotten it. You know that.”

His boss nodded his head and got into thoughtful contemplation for a few
seconds. He then looked at Guru and said, “Guru, I have an idea—what stops you
from contesting the ward councillor election yourself? I am sure all our workers
will support you. Since you live in this area, and know almost everybody in
the colony, you should be able to convince people to vote for you. You have a
good name among them. What do you think?”

Guru said, “But I have never contested an election before, Sir. Besides, I don't
have the money required for campaigning.”

His boss said, “Don't worry about funds. I'll sponsor your campaign. It's time
you stop slogging as a labourer and start handling bigger responsibilities. A
journey of a thousand kilometres begins with a single step. Go ahead and file your
nomination papers—you have my blessings.” Guru didn't anticipate how big that
responsibility might become, one day.

It was then that Guru visited the Tirupati Temple for the first time. A day before the
election results were announced, he became nervous and someone suggested he
visit the powerful God of the Seven Hills in Tirupati. To overcome his anxiety,
Guru boarded the earliest bus to Tirupati, even though he did not have the
habit of visiting temples.

As soon as he reached Tirupati, he walked to Alipiri and then made his way
up by foot instead of taking another bus to reach the hilltop Tirumala temple.
He reached the temple at midnight, and he had to wait for twelve hours for
the darshan of the God. Even though the journey was physically exhausting, it
soothed his nerves and helped him shed the mental tension of the 
previous day. After coming out of the temple, Guru was much more relaxed. He
knew that the result would have been announced by then. He was tempted to call
home, but instead, he bought food, distributed it among the hungry poor, ate some
and boarded a direct bus to Chennai.

He reached Chennai in the evening and went home directly. When he knocked on the
door, his
wife rushed to open it, beamed, and informed him that he had won the election. He
embraced her, looked into her eyes, and said calmly, “I know.”

Thereafter, he made it a point to visit the Tirupati temple at least once a
year. Thirty-five years passed since then. From being a ward councillor, Guru
became a member of the Legislative Assembly and a state minister, and later,
a member of the parliament and a central minister. During his second term as a central
minister he managed to wrest the 'plum' Ministry of Power portfolio from a rival.
The rival minister, however, convinced the prime minister to split it
into two and became in charge of the Renewable Energy portfolio.

This was a major setback for Guru as new power projects were about to be
sanctioned to produce power using renewable energy technologies, and additional
funds were released to subsidize the rooftop and commercial solar installations.
The renewable energy projects were completed much faster compared to the
fossil fuel and nuclear-based power plants he was setting up. There were plenty of
business opportunities in the Ministry of Renewable Energy, and all of them were
snatched away from him.

Guru did not take this defeat lightly. Not only was he addicted to money and
power, but he also wanted to be invincible. For the first time, he chose not to
circumvent the roadblocks in his path—he wanted to uproot them, reduce them to
rubble, and trample triumphantly over them, turning them to dust. That, he thought,
would be a fitting reply to his rival.
