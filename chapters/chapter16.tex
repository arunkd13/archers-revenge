\chapter{}

\lettrine{I}{t was} a few days since Divya moved into Aryan's forest house. Initially, she
found it difficult to adjust to the natural surroundings with minimal amenities,
but eventually she got used to living alone in the forest. She had to do much
more physical work than in a city, but all those efforts helped her remain fit
and agile. She felt more active and energetic in her new house than any air
conditioned room she had lived in. She wondered if people were never designed to
live in comfortable houses and follow a sedentary lifestyle.

One day, while at lunch prepared by Divya, Aryan said, “I think we need to plan
our next attack.”

“Do you have any plan?” Divya asked. “The Minister is not going to come to
Tirupati again for a while.”

Aryan said, “He is not going to come to Tirupati anytime soon, but he will be in
Chennai during the next month. Elections are around the corner and he has a busy
campaigning schedule in the city.”

Divya said, “Chennai is quite close – just 160 KM from here.”

Aryan said, “Yes. He will be starting his election campaign by throwing a party
for few influential people at his guest house located in the outskirts of the
city. I have located the place on Google Maps, but I don't know how or from
where we are going to attack. The security cover will be tight, especially in
the wake of our recent attacks.”

Divya said, “Seems like a good opportunity. Let's attack him during this party.”

Aryan said, “How? We can't stop his car while coming in or going out - I am sure
security will escort him on the front and rear. We cannot get into the house
with weapons. In fact, we cannot get anywhere close to his vicinity – they will
check everyone moving in and around. How can we attack him?”

Divya said, “As I see it, we cannot get in close range or get a direct line of
sight. Is that the problem?”

Aryan said, “Yes, how can we use our bows and arrows to attack him in such a
scenario?”

Divya said, “According to the archery training manual, it's not possible. But
archery doesn't always require a line of sight and archers have attacked their
enemies from far away, even if there was a huge obstacle - like a wall - in
between. That's why it pays to know your history and think a little bit out of
the box. In your case, out of the practice manual.”

Aryan said, “That's impossible. How can you attack somebody from far away, when
there is a large obstacle in between you and them? Doesn't that mean you can't
see them and there is no way to take aim? How will you launch the arrows then?
Towards the sky?” he smiled.

Divya replied in a serious tone, “Yes, towards the sky. Think about this – once
you launch an arrow towards the sky, it will travel up for some distance, lose
velocity and come back towards the earth after a short while – right? Unlike
bullets which are released with a great force, arrows travel much slower. Hence,
it is possible to estimate where they will fall down.”

“So?” Aryan said.

“If we estimate the angle and force of release of an arrow from a bow, we can
easily determine where it will fall down on the earth. And it will fall down
with the sharp edge facing towards the earth. That means, even if there is a
huge wall in front of us, we can still strike an enemy standing on the other
side of the wall by adjusting the force and angle of release, provided we know
where he is standing.” Divya said.

“What do you propose we do in our case?” Aryan asked.

Divya said, “Simple! We know he is going to be on the terrace along with his
guests. We can position ourselves around 60 – 80 meters away from the spot and
launch our arrows in a sloping angle, towards the sky. They will go up
initially, gain some distance, and come down to hit our target area after a
short time. But we need to fix the angle of elevation, force of release, and
practice it until our delivery is consistent.”

Aryan said, “That is absurd. Madam, please remember that even if we have a clear
line of sight, hitting the target requires a lot of accuracy, force, and
practice. It's easy to miss the target even in such a simple scenario. I think
we should focus on how to get into the line of sight and get a clear view of the
minister so that we can strike right at his heart.”

Divya said, “You just said that the Minister, especially in the wake of the
recent attacks, will be having many armed security guards around him. Their guns
are far more accurate and faster than our arrows, I guess?”

Aryan nodded his head in affirmation.

Divya continued, “So, if we get anywhere within their line of sight, the chances
that they will shoot us before we've even launched a single arrow is very good.
Am I wrong?”

Aryan said, “No, you are right. We can't afford to do that.”

Divya said, “In such a situation, getting into their line of sight would only
make it easier for them to kill us. It's almost a suicidal move for us. But,
think about this – what advantage do we have that they don't?”

Aryan was silent.

Divya said, “Our arrows maybe slower than their bullets, but in this situation
we are going to convert that disadvantage into an advantage. We are going to
attack from a safe distance and escape from there before they even realize what
happened. We should engage our enemy on our terms and strengths - not theirs.”

Aryan said, “Even if I agree to your suggestion in principle, how can you ensure
accuracy? There is no way you can aim and hit one particular person using this
method.”

Divya said, “Our aim is not to kill the Minister right away. The Greeks used
this technique to hurt and intimidate their opposition just before a battle. In
other words, they used archery to mentally weaken their opposition so that their
cavalry and artillery can finish the job. The opposition back then defended
themselves using large shields but our opposition doesn't have any of that. Our
aim through this campaign should be to create fear in his mind and embarrass him
in front of his guests. We need to shake the ground beneath him so that he
doesn't feel secure anywhere, anytime. People who are afraid will make mistakes.
We will wait and take advantage of it.”

Aryan said, “But there will be other people in the crowd. Our arrows may fall
anywhere on the terrace or even around the terrace. What if others are killed?”

Divya said, “We will use blunt arrows. People may get hurt, but there will not
be any serious wounds.”

Aryan said, “Not bad. We spoil his party, embarrass him before his guests, and
if we are lucky, one of the arrows might even hit and hurt him. We will convey a
clear message to him that his killers are closely watching him and will keep
striking at him where ever he is. That will make him live in constant fear of
his life and will make him think twice before going anywhere outside. He need to
realize that life is not a commodity to be taken at will. Let's do this.”

Since the Minister's guest house was close to the beach, Aryan and Divya
identified the exact spot from where they would launch their arrows (with the
help of Google Maps). There were two more houses in front of the Minister's
house and then there was a short lane, followed by the beach. Their chosen
location was on the beach, around 60m away from the Minister's guest house – a
distance that was perfect for their mode of attacking.

Over the next few days, they marked an equivalent distance between the actual
target area and their position, on the hills. They practiced the precise angle
to hold their bow and the force with which they had to launch their arrows in
order to hit their target area, which covered the same length and breadth as the
terrace of the Minister's guest house. Accuracy depended on the angle of
elevation of their bow and the distance over which their bow-string was
stretched (force of release). They practiced repeatedly in order to achieve a
fairly consistent result to ensure their arrows fall somewhere within the marked
area.