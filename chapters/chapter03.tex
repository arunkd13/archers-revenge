\chapter{}

Aryan was two when his mother passed away. His father brought him up
single-handedly ever since. His father did not remarry as he was not sure how a
step-mother might treat Aryan. There was no help from the extended family as
they were all based in Kanpur, not New Delhi, where Aryan grew up.

Some of Aryan's earliest memories were of his school and home. His father woke
him up early in the morning, cooked and packed lunch, dropped him at school,
picked and left him at home after school, went back to the office, came home at
late-evening, and helped Aryan with his studies. Most of the days he slept
early, with Aryan.

Aryan's father was doing the work of his mother too, and Aryan couldn't
recollect anything that suggested his father was tired of bringing him up. He
was insistent on providing the best possible life to his son and Aryan
reciprocated by being committed to his tasks – a trait he picked up from his
father. Aryan pursued even the most difficult goals with ardent determination.

While most of Aryan's friends opted for tuition classes after school, Aryan did
not. His father decided Aryan would get the best possible tutoring from him. He
thought tuition classes added very little value. Perhaps him working as an
Engineer in the Government Department made things easy for him, but Aryan was
sure his father gave his full efforts at work too.

There was one important activity Aryan was involved from a very young age –
Archery.

Aryan's father wanted Aryan to bring home an Olympic medal. And archery was the
sport he chose for Aryan. From an early age, Aryan started his training under a
professional coach at the local archery club. Initially, Archery was not special
to Aryan – he considered it as another subject to score, another game to win.
But since Aryan was dedicated to learning the skill, he practiced longer than
anyone else. As a result, his accuracy improved greatly. Good coaching and
Aryan's dedication were important, but all those hours spent by his father in
dropping and bringing him back home, especially when Aryan was younger, was
important in making him a world-class archer.

Aryan gradually developed a liking towards archery – all his victories and
awards helped. Aryan was better than an average archer of his age and his
performance was reasonably good in the archery competitions. But he was not sure
if he could win an Olympic medal. His father, however, never allowed Aryan to
lose hope - he insisted that Aryan focus on learning the skill and not bother
about results. Aryan's father was an idealist who knew how to transform
intentions into reality through concrete actions and motivation. Aryan may or may
not win an Olympic medal, but it was important to his father that Aryan gave his
best shot at it.

When Aryan's father came to know about the Environmental Engineering course in
UK which also had a world-class archery academy, he immediately decided to admit
Aryan there. He thought it would give Aryan much needed international exposure
to the sport and also provide a firm footing academically.

Besides, the newly formed Renewable Energy Department was looking for people at
Executive level positions and Aryan's father was selected as a top-level
executive. He was finally able to dedicate his time to the field he believed
could make a huge difference to India's energy security, and enable electricity
to reach the masses.

Aryan knew his father's new job at the Renewable Energy Department was no
accident. His father had always been passionate about the environment and clean
energy. Aryan could recollect the passion with which his father used to discuss
with him about decentralization of energy market, enabling people to generate
their own power, taking electricity to the remotest corner of India, reducing
pollution, thwarting global warming, and how renewable energy played a
significant role in enabling all that. These discussions, in part, inspired
Aryan to study environmental engineering and his father was very happy.