\chapter{}

\lettrine{D}{uring} the next two weeks, Aryan and Divya planned their attack
down to the last detail and rehearsed it many times. This was going to be their
last chance, and they were determined to give it their best.

The appointed day arrived. Aryan and Divya positioned themselves on the same
elevated mountain patch between the U-turns, from where they had attacked Guru a
few months earlier—albeit independently. The memories of the failed attack on
that day still haunted Aryan. He wanted to conquer his fear of failure by
successfully executing his attack this time. This was another reason why he
chose the same spot.

Though both Aryan and Divya were on the same elevated mountain patch, they were
on opposite sides. Divya positioned herself on the side where she could
see the vehicles approach the U-turn. Aryan positioned himself on the other side,
from where he could see the vehicles leave the U-turn.  Both of them could see
vehicles passing below as the mountain patch was just four meters above the
road. Since there were some trees, rocks, and vegetation, they couldn't see each
other, but they held on to two ends of a single long rope as they waited.

Divya's job was to pull the rope towards herself immediately after she saw
Guru's car, a white Ambassador with flashing lights on the top. This action
would indicate the arrival of Guru at the U-turn. Aryan needed the signalling
for a reason. He was holding an arrow in one hand and a matchbox in the other.
The arrowhead on this arrow was wrapped in a cloth dipped in kerosene. The match
box was to light the cloth wrapped on the arrowhead so that it caught fire. He
had kept his bow on the ground next to him and had a few poison-soaked arrows in
a large pouch tied to his back. From his position, he could see the road, and
the vehicles approaching him once they crossed the U-turn.

Opposite Aryan, just before the short protective boundary wall, there was a heap
of sand. This was no ordinary heap of sand—it had been placed there for a
reason. The heap was about four feet high at the narrow top and had a circular
base about three feet in diameter. Such heaps of sand are normally found near
construction sites, but at this location it served another purpose altogether.

Inside the heap of sand, Aryan and Divya had placed an earthen pot filled with
petrol on the previous night. The opening of the pot was placed facing Aryan and
was covered with a thin felt paper. A loose layer of sand covered the felt paper
to hide the mouth of the pot from onlookers. The setup looked just like a heap
of sand, but Aryan knew exactly where he had to launch an arrow with a burning
arrowhead in order to create an explosion.

Yes, they were going to create an explosion!

They had replicated the setup a few times in the forest and were satisfied with
the explosion that ensued. There was a small lag of about 1.5 seconds from the
time the burning arrow went into the pot to the time the explosion actually
occurred. This necessitated early spotting of the car and that in turn required
Divya to alert Aryan once she spotted the car. They had marked a line on the
road, and Aryan was supposed to release the arrow when the front tires of the
car crossed the line. This would ensure the car was close to the heap when the
explosion occurred.

Aryan felt the rope tighten in his hands. The next instant, Divya pulled away
the rope in one swift movement. That was the signal indicating the approach of
Guru's car towards the U-turn. Aryan took a matchstick, created a fire and
lit the kerosene-soaked cloth on his arrowhead. He then placed the arrow
with the burning arrowhead on his bow. He pulled the bowstring, got into
the attack position, and waited as he looked towards his left. The white
Ambassador car turned around the U-turn and was now moving towards him. He was
ready to release the arrow, but suddenly, the car slowed down and stopped just
short of the line.

The glass windows were open, and he could see a lady in the back seat. Probably,
it was Guru's wife. Aryan remembered that Guru had come to the temple to
celebrate his wedding anniversary. A hand was pointing towards a location that
was about six to eight meters in front of where he was standing and everyone in
the car was looking out. Aryan quickly lowered his flaming arrow to hide it
from their sight but realized they were not looking at him. Perhaps Guru was
narrating the event of his attack a few months back. They did not seem to notice
Aryan, as he was hiding behind a tree. By now only his face had popped out to monitor
them.

The car had stopped for two minutes and by this time Divya had come over from the
other side.

“What happened? Why didn't you shoot the arrow yet?” she asked.

“They've stopped the car. But they'll have to move it soon—other vehicles
behind won't wait forever. You go to your position,” he said.

Divya went and stood ten metres ahead of Aryan where they had tied a rope. The
rope was tied around a tree and the open end reached up to the road below. They
had parked their bike on one corner of the road ahead of their location. She was
ready to get down and bring the bike immediately after the explosion. This was
crucial for them to escape from that location.

After another long and agonizing minute, the car at last started moving forward.
Aryan got back into the striking position, along with his burning arrowhead. As the
car crossed the line, Aryan fired his arrow. Within the next one second, it flew
towards the heap of sand and pierced into the felt paper on the pot. The arrow
delivered the flame straight into the pot containing petrol. That set off a
huge explosion, and the flames spread to almost half the road. Even though the
driver saw the explosion before him, it happened so quickly that he couldn't
apply brakes.

The car moved through the flames and smoke. Some petrol was splashed
on the left side of the car during the explosion, and flames soon engulfed the
car. The fire spread inside the car too. The car crossed the explosion site, but
the driver lost control and hit the boundary wall and the car stopped. The two
right-side doors of the car opened and both the driver and the lady got off the
car puffing due to the fire and smoke. They pulled Guru and his security guard out
of the car. Some parts of their clothes were still burning. Aryan was able to
see Guru moving—he was still alive.

Aryan anticipated this possibility—that's why he had brought poison-soaked
arrows with him. He quickly dropped one side of a rope on the road. The other
side was already tied to a tree. He used it to climb down to the road and walked
towards Guru, who was on his wife's lap. Aryan moved forward until he was just
five feet away from Guru. He took an arrow, placed it on the bow string,
stretched it back, and aimed at Guru's heart. He took one deep breath.

Aryan turned his head impulsively and looked back. Divya was standing ten metres
behind him on the bike. She was looking at him silently. He looked at the
minister once again; his eyes were filled with tears.

Aryan raised his bow upwards, looked at the sky above him, and released the
arrow towards the sky. He couldn't do it. He couldn't bring himself to kill
Guru.

He turned back and walked towards Divya. They got on the bike and rode downhill
as fast as they could.