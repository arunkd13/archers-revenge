\chapter{}

\lettrine{D}{ivya}, 20, was born and brought up in Chennai, the capital city of Tamil Nadu.
Despite being the eldest child in a family of three, she was pampered more than
her younger brothers by her parents. Her brothers were forced to study and had
very strict schedules. There was lot of pressure on them to gain and maintain a
top rank in academics.

On the other hand, Divya was a free person. She just couldn't get herself to do
anything against her liking. She was not academically weak, but she was not
excellent either. She forced herself to memorize the notes just before exams so
that she did not fail. Otherwise, she did what she wanted to and her parents did
not discourage her activities. If she wanted to participate in some activity,
her parents never refused. If she didn't want to join any activity, her parents
did not force her. In comparison, her brothers' schedule was planned to the last
minute.

She didn't consider this behavior of her parents as gender discrimination. She
was actually scared that her parents might control her too, but fortunately,
they didn't. Maybe they tried and they were not able to, but she was thankful
that she was not forced to do anything. That freedom was sufficient for her to
love her parents dearly.

From a young age, Divya showed keen interest in archery. Her parents enrolled
her for music classes - which lasted no more than fifteen days. During the
course, she noticed children practicing archery in the neighborhood compound.
The more she watched the game, the more she liked it. It was almost love at
first sight, and the intensity of her admiration towards archery solidified
further during the time she was supposed to be singing. She was fascinated by
bows and arrows.

In the next few days, she made her own bow and arrows using broken sticks from
the tree and created havoc at home by breaking glasses and cutlery with her
arrows. Unable to take any more damage, her parents made her join an archery
academy. Even though she hated the routine and structure of the archery coaching
class, for once she did what she was told as she didn't want to be sent out of
this one.

Every time she hit the target accurately, her joy abounded. Very quickly, she
became the dear student of the instructors and was granted several privileges
like extra practice time. Archery was not just a sport but was also her best
companion. She learned what the instructors taught her, but also learned many
new things about archery from various books published on the subject. The
proximity of her home to Connemera Library, – one of the four national
depository libraries that got a copy of all books published in India – helped.

She participated and won a few local competitions but she never participated in
district-level, state-level or national-level contests, in spite of her
instructors insisting on it. She did not like participating in archery contests.
Archery was her passion and she did not want anyone's approval or rewards or
rankings to prove anything. She viewed positions and rankings as being
superficial. Her passion for the art of archery was pure and she didn't want to
adulterate it by running after prizes and medals.