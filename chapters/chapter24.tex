\chapter{}

Divya hired an auto and was at home within the next ten minutes. She was glad to
be back in the familiar surroundings. She realized how much she missed home only
after she reached there. She opened the main gate, went inside, and rang the
bell on the second gate that led to the living room. Her mother opened the door.

“Why did you call me now, mom? You know I cannot be roaming around here,” she
said.

“There is an important event today evening and you'll have to be present,” her
mother said.

What could be so important, Divya wondered.

“We have fixed a match for you. People from the boy's house are coming to see
you today evening,” her mother said.

“You are planning my marriage . . . without even asking me?” Divya asked.

“That's how it has always been in our family. Elders know what's good for you
and youngsters are expected to obey,” her mother said.

“What do you mean I'll have to do whatever you say? There is no way I am getting
married right now. There are so many things I need to do before that,” Divya
protested.

“You can do whatever you want but after the marriage,” her mother said.

“You very well know what I am doing now. In spite of that-” Divya said.

“Yes. I know what you are doing now. You are staying alone with some guy on the
hills,” her mother said.

“What? Is that what's bothering you? Don't you trust me?” Divya screamed.

Her mother was silent and looked away indicating she was in no mood to change
her decision. At that moment, Divya's grandmother walked in from the next room.
On seeing her grandmother Divya was much relieved. She went and hugged her. “See
what mother is trying to do grandma. She is trying to get me married without
my-”

“We'll talk about that later,” her grandmother patted on Divya's back and asked
her to sit down on the sofa. “First, let's drink some tea. Do you want to have
tea?” Her grandmother motioned to her mother to get some tea from the kitchen.

“No, I don't want tea,” Divya said.

“Why don't you want tea?” her grandmother asked.

“I . . . I don't feel like drinking tea now,” Divya said.

“Are you not used to drinking tea?” her grandmother asked.

“Of course I am used to drinking tea. You know that very well,” Divya said.

“Then why are you refusing?” her grandmother asked.

“I told you . . . I don't feel like having it now,” Divya said.

“What do you mean? Unless you don't like it or you are not used to it, there is
no reason for you to refuse tea,” her grandmother said.

“What the . . . Wait. I don't want tea because my body is heated up and I have
some mouth blisters. Having tea now will only increase it,” Divya said.

“You know what - tea is good for people with body heat and mouth blisters,” her
grandmother assured her.

“What??? This is totally unscientific and-”, Divya protested.

“You were born two generations after me. I know what is good and what is not.
Don't use science as an excuse,” her grandmother said.

“Can't I decide whether I want to drink tea or not? Don't I have even that
freedom?” Divya was almost in tears.

“Yes, only you will decide. Now tell me - do you want tea?” her grandmother
asked.

Divya realized that there was no point in arguing with her grandmother who had
suddenly become irrational and adamant, today. She agreed to have tea for the
sake of it. Once the tea was served, Divya was about to sip the tea when her
grandmother asked,

“Now tell me - do you want to get married?”

The reason behind the silly tea politics was now evident to Divya. She threw the
tea-cup on the floor, went to her room and locked herself up. She would sleep
for sometime to remain protected from the marriage-coercers outside.

Divya slept for a couple of hours when she heard her mother knocking on the
door. She didn't bother about it for a few seconds but her mother was in no mood
to relent. Divya woke up from the bed, walked slowly towards the door and opened
it reluctantly.

“What took you so long to open the door?” her mother asked after she came inside
the room.

“I was sleeping mom. I am tired – please let me sleep for some more time,” Divya
said.

“Nonsense. The boy's family has already come and you are not even dressed. Here,
wear this Kanchipuram silk sari and get ready in another ten minutes. In the
mean time, we'll keep talking to them to keep them engaged. Make it fast,” her
mother said and left in haste.

Divya decided to put an end to this marriage business. She knew her family
wouldn't listen to her. But if the guy refused to marry her, surely his family
would listen to him? She dressed up quickly and was ready for the girl-seeing
ritual.

Ten minutes later, her mother came in, handed a tray full of cups of tea and
asked her to give it to the boy's family. Divya served tea to everyone.

“Divya made this tea herself. She also cooks well,” her mother said to all.
Suddenly, the boy's mother started wailing in pain. Everyone looked at her.
Divya, who was serving tea to her, said “Oh no - I am so sorry Aunty. I stepped
on your feet accidentally.” The boy's mother was definitely not amused but she
gave a fake smile to mean she had not taken the incident “sportively”.

Since the boy felt that they had to “thoroughly understand” each other before
marriage, both of them were ushered into Divya's room so that they could talk to
each other in private. Divya's mother whispered into her ears just before she
entered the room, “Keep looking down and let him talk. Don't talk unless you
need to – is that clear?” Divya nodded and went inside. Divya sat on her bed
while the boy sat on a chair opposite to her.

He looked at her and smiled. She grinned from ear to ear for a couple of seconds
and immediately shifted to her serious look.

“Your smile is sweet,” he blushed.

He found that smile sweet? This guy was going to be difficult to break.

“My mother told me that you have completed your B.Sc. I hope you are fine with
being a homemaker after marriage. Do you want to take up a job?” he asked.

“Why? Don't you prefer girls who are independent? Would you not allow me to work
after marriage?” she asked.

“I am modern and I don't mind you taking up a job. But my mother is traditional
– she doesn't prefer her daughter-in-law working,” he said.

“So, you'll listen to whatever your mother says even after marriage. You won't
take your own decisions,” she said.

“Not like that, but we can't totally disregard what elders say, no?” he said.

“First of all, why should they interfere with our life? I prefer a nuclear
family – we'll rent a house and live separately,” she said.

“But your mother said you're fine with a joint family,” he said.

“How can my mother decide for me? My opinions are mine, her opinions are hers,”
she said.

He didn't like the way the conversation was heading. He thought maybe he should
try a different topic. “What are your hobbies?” he asked.

“Shopping. I love shopping for gold jewelery and silk saris. I buy one of these
two items every alternate week,” she said looking straight into his eyes.
“But they are so expensive, aren't they?” he said.

“Yes. Actually my father has run out of money. That's why he wants to get me
married – so that my husband can take up the burden. Er- I mean, pleasure,” she
said and smiled mockingly once again.

He didn't know why his questions were met with such stern answers. Maybe he had
to learn the art of conversing with women. Time for changing the topic once
again.

“You know what, I love adventure sports. I once trekked to the top of a hill,
camped and returned the next day,” he said and beamed.

“What's the big deal with that? I've been living on the top of a hill, amidst
the forest, over the last two months,” she said.

His eyes widened. “How. . . how did you manage that by yourselves?”

“Did I say I was alone? I lived there with a guy. My boyfriend, actually,” she
said.

By now, his mouth was wide open. “I . . . I think my mother is calling me. We'll
meet sometime later,” he said and went out hastily.

She knew her goal was accomplished.

