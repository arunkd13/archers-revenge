\chapter{}

\lettrine{T}{he} next day, Divya left to Tirupati from Chennai by train. She used to love
train journeys—it brought back memories of her yearly childhood trips. Like
then, she sat at her favourite window seat and indulged in her favourite
activity—getting a glimpse of nature, civilization, and people. As her journey
progressed, she couldn't concentrate outside. She was totally preoccupied with
her inner thoughts and reflections. What was she doing? Was she going in the
right path? Playing with other's lives—is that the correct thing to do? How
long will she have to continue with this double game? What would she gain by
playing this dangerous game? Will all this trouble be worth in the end?

There were so many questions ringing in her mind, but answers were not as
forthcoming. She now realized how simple those questions at college exams
were—at least there were clear answers. The exam of life, however, seemed to be
much more complex. In the path she had chosen, lives were at stake. She didn't know
if she was strong enough to deal with the consequences of her actions. But she
realized she didn't have options before her now—she had already chosen her
path and will have to complete the task at hand. Rest, she will decide later on.
Everything needed thorough reconsideration.

She was immersed in her thoughts that the train was already at Tirupati railway
station. Three hours of journey felt like three minutes. Divya got down at the
station and was crossing the main gate when she saw two ladies in churidar
getting down from a jeep. They looked in her direction for a couple of seconds
and walked towards her. Only when they were a couple of feet away, it hit her
that they could be policewomen in plain-clothes. It was too late to escape.
Before Divya could react, they held Divya by her arms from either side and
dragged her towards the jeep.

“Who are you? Why are you dragging me?” Divya protested and tried to free her
arms. But they were stronger and the grip was firm.

“You need to come with us to the police station,” the woman replied.

“Why should I? Do you have an arrest warrant?” She asked.

“We don't need a warrant to question people. Just keep quiet and come with us,
otherwise you'll invite more trouble,” the other woman replied and handcuffed
her from behind.

“Why are you handcuffing me? Just remove them now or—” Divya shouted.

They were in no mood to listen to her, continued dragging her towards the jeep
and put her in. There were two men constables in uniform waiting inside the
jeep. They drove her to the police station. Divya remembered Aryan's suggestion
not to use public transport or roam around in public places—she cursed herself
for forgetting that crucial bit of advice. Upon reaching the police station,
they took her to one of the cells, tied both her hands behind a chair, switched
off the lights and locked the door from outside. She tried calling them, but
there was no reply. Unlike other cells that had bars and one could look outside,
this one had an opaque door. After a short time, Divya became tired of shouting
and fell asleep on the chair in the interrogation room.

Divya woke up as water splashed on her face. She had no idea how long she had
slept. The lights had been switched on, and she was able to see the two lady
constables in front of her. Both of them held a bucket with water. One bucket
was empty—probably that was the one just emptied on her, she thought, feeling
the water dripping from her face. The second constable now stepped forward and
splashed another bucket of water on Divya's face with great force. Divya jerked
back and couldn't see anything for a couple of seconds until the water dripped
away. The inspector walked in and asked the women to leave.

“Who are you and why are you trying to kill the minister?” the inspector asked.

Divya turned her head sideways in defiance, not wanting to answer.

The inspector slapped her. Since she was never beaten by anyone, not even by her
parents, Divya felt a shock-wave pass through her body. The pain of an
experienced police inspector slapping her, was too much. She turned towards him
and opened her mouth to say something.

He slapped her again. This time his fingers hit her lips which collided with her
teeth. The force of the impact cut her lips and blood started flowing from it.
She broke down and her eyes were filled with tears.

The inspector allowed a few seconds to pass. And then he said, “Either if you be
silent or if you lie while answering my questions, I'll slap you twice as hard
as this—is that clear?” He raised his voice while uttering those last three
words.

She nodded her head.

The inspector asked, “Who are you and why are you trying to attack the
minister?”

Divya replied, “I didn't want to attack the minister. My boyfriend attacked him
because the minister killed his father. He wants to avenge the death of his
father—I was just helping him.”

“By just helping him, you have attempted murder. Do you have any idea how many
years you two will be locked up inside the prison?” the inspector asked.

Divya was silent.

“What are your names? Where is he now?” the inspector asked.

“My name is Divya. His name is Aryan. He lives in a small house in the forest in
the Tirupati hills. Please leave me—I am innocent. It was Aryan who planned
and executed everything. I was just doing what he asked me to. Please let me
go,” she pleaded with tearful eyes.

“OK. We'll let you go,” the inspector said after a brief pause.

\fancybreak{* * *}

Aryan was surfing the net when Divya entered the makeshift home. He saw Divya
standing next to the door. He beamed and said, “Welcome home. Good to see you
once again. Did you notice the newspapers, TV channels and news sites? All of
them are talking about our attack on the other day. I am sure it was a huge blow
to the minister especially on the election eve. I think we have succeeded in our
efforts to connect the minister with the murders. Reporters are already
investigating that line. I hope the police will also start investigating soon
and arrest him.”

Divya was standing quietly and did not reply.

How did you come?” Aryan asked and looked at the computer.

“I came by train,” she said and sat down on the floor.

“You came by train? I told you not to travel by bus or train—it's risky. The
police maybe monitoring those places by now,” he said.

She didn't reply. Her silence made him feel jittery. He turned behind and saw
her sitting down. She had covered her face with her hands. He got up from the
chair, went to her and removed her hands from her face. Her eyes were filled
with tears, and she did not look up.

“What happened Divya?” he asked. She was still silent.

Aryan noticed some movement, and he turned to look outside the door. There were
at least six policemen standing with their rifles pointed towards him. He
realized the gravity of the situation as a chill went down his spine.
