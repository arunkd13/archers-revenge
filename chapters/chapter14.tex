\chapter{}

\lettrine{E}{ven} though Guru, the minister, had initiated investigations through the
police, he rightly guessed that his personal investigations might be more useful
in identifying the suspects in this case. Only he knew his enemies—and, more
importantly, his crimes.

Initially, it was difficult for him to guess who might have been behind the
assassination attempt. Not that he had no enemies—in his profession, everyone
was a friend one day and an enemy the next. People frequently oscillated
between friendship and enmity, based on their commercial interests and Guru's
position. Hence, any of his “associates” or “well-wishers” could have been behind
the assassination attempt. Except in rare cases, even in his profession
of politics, people didn't kill others due to commercial losses or power struggles. Besides, it was difficult to get away with murder. Hence, he
ruled out professional and commercial motivations.

He thought about all the recent crimes he had committed. All of them were
related to money, except one from a few months ago, when he killed three people to
take control over the Ministry of Renewable Energy. A personal loss, of
a loved one, could be a powerful motivation for revenge.

He asked his people to conduct some preliminary investigations on the immediate
family members of the people he had killed. At least in one
case—Aryan's—there was sufficient cause for suspicion. Aryan had tried to register
a police
complaint and the minister had sent his men to silence him. One of his men
contacted the University where Aryan was studying and was informed that Aryan
had applied for a long leave on health grounds. His apartment at New Delhi had
been locked for the last couple of months and the neighbours hadn't seen him
around during that time. No one knew where he was.

The minister wondered if he should have killed Aryan. But there were no signs of
any violence from that kid back then—he had made sure Aryan was sufficiently
threatened. The minister also discovered Aryan's background in archery training
and was able to make the connection with the way he had been attacked.

Inspector Samara Simha Reddy had already arranged a thorough check of the
spot where the murder was attempted. Apart from the nail cluster which was used to
puncture the tires of the car and a couple of arrows, they could not recover
anything else. There were no fingerprints on either of the objects—whoever
attacked was probably wearing gloves. There was no body, blood, or any other clue
that could lead to the suspects.

Just as the inspector was pondering over how to proceed with the case, the
minister called him and informed him about the probable suspect. But the
minister didn't tell him about the context of the crime or why Aryan was trying
to kill him. It didn't take long for the inspector to figure it out. In India,
it may be possible to escape conviction for a crime, if one had enough
money and influence or at least delay the judicial process. But it is impossible
to stop the truth from circulating. The inspector only had to talk to a
couple of his friends and their contacts in Delhi and do a little research to
establish the circumstances and motivations behind the crime. What happened now
was an attempt to take revenge on the minister for the murders committed by him a few
months back. Now that the inspector knew the truth, he could use it to demand
more payment from the minister, eventually.