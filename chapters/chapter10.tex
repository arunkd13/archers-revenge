\chapter{}

\lettrine{D}{ivya}, 20, was born and brought up in Chennai, the capital city of Tamil Nadu.
Despite being the eldest child in a family of three, her parents pampered her more than
her younger brothers. Her brothers, on the other hand, were forced to study under
very strict schedules. There was immense pressure on them to excel academically and maintain a top rank.

Divya was always a free-spirited person. She just couldn't get herself to do
anything against her liking. She was not academically weak, but she never tried to excel. She forced herself to memorize the notes just before exams so
that she could just pass. Otherwise, she did what she wanted and her parents did
not discourage her activities. If she wanted to participate in some activity,
her parents never refused. If she didn't want to join any activity, her parents
did not force her. In contrast, her brothers' schedules were meticulously planned,
down to the last minute.

She didn't consider this behaviour of her parents as gender discrimination. She
was actually scared that her parents might start controlling her too, but, fortunately,
they didn't. Perhaps they tried, but were unsuccessful; nonetheless, she was thankful
that she was not forced to do anything against her wish. That freedom was sufficient
for her to love her parents dearly.

From a young age, Divya showed a keen interest in archery. Her parents enrolled
her in music classes—which lasted only fifteen days. During the
course, she noticed children practising archery in the neighbouring compound.
The more she watched the game, the more she liked it. It was almost love at
first sight, and the intensity of her admiration solidified
further during the time she was supposed to be singing. She was fascinated by
bows and arrows.

In the next few days, she made her own bow and a set of arrows, using broken
sticks from a tree, and created havoc at home by breaking glasses and
cutlery with her practice sessions. Unable to take any more damage, her parents made her join an archery academy. Even though she hated the strict routine and structure of the archery coaching
class, for once she did what she was told, because she didn't want to be thrown out of
this one.

Every time she hit the target accurately, her joy abounded. Very quickly, she
became the dear student of the instructors and was granted several privileges
like extra practice time. Archery was not just a sport but was also her best
companion. She learned what the instructors taught her, but also learned many
new things about archery from various books published on the subject. The
proximity of her home to the Connemara Library—one of the four national
depository libraries that received a copy of all books published in India—helped.

She participated and won a few local competitions, but she never participated in the
district, state, or national level, in spite of her
instructors insisting on it. She did not like participating in archery contests.
Archery was her passion, and she did not want anyone's approval or rewards or
rankings to prove anything. She viewed positions and rankings as superficial. Her
passion for the art of archery was pure, and she didn't want to
adulterate it by running after prizes and medals.