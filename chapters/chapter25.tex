\chapter{}

\lettrine{T}{wo} weeks after they were arrested, the general elections got over.
The minister found time to visit Aryan and Divya in prison. They were brought to
the interrogation room from their respective cells. Three chairs had been placed
at equal distances around a circular table. Aryan and Divya were seated on two
of them. The other chair was empty. They saw each other but didn't speak.

After a couple of minutes, the door opened, and the minister walked in. He sat
on his chair and looked at both of them.

“I am sure the jail food is better than the forest food you were eating over the
last couple of months”, the minister said, laughing to himself. No one thought
it was a good joke, except for the minister.

After laughing for a few seconds, the minister suddenly became serious and said,
“What did you two think? You could kill me with a couple of bows and arrows? You
thought you could use this outdated archery to fight against me? You two were
not only technologically challenged but also heavily outnumbered. I
have the entire police department under my control.”

He paused and continued, “However, I give you credit for evading arrest all this
while. We should have caught you a long time ago, but you two managed to stay
out of our reach. It was a good idea—living in the forest on the
hills.”

Aryan and Divya had their gazes fixed on the table.

“You both are almost as old as my own children. I can't even imagine youngsters
like you spoiling your lives with impractical ideas and outdated methods. You
could have at least tried getting guns—you'd have had a better chance of killing
me. But I guess you don't know where to buy them”, the minister said.

There was no reaction from either of them.

The minister turned towards Aryan and said, “I was open to doing business with your
father, but he did not cooperate. For him, his cause—this renewable energy—was
more important than business. If he had been practical and realistic, he would
have been very rich and more importantly living with you today. He proved to be a major
obstruction to my plans, and I had no other option but to kill him, along with
the other two people. I see the same diffidence in your eyes that I saw in your
father's. You should understand that people live to do business and make money,
not to further some imaginary causes.”

Aryan looked straight into the minister's eyes and said, “We should not have
used those rubber tips on our arrowheads while attacking you during the meeting.
You would have most certainly become history by now.”

“So, why did you?” the minister asked.

“We believe in the legal system. We wanted to give justice a chance”, Aryan
replied.

The minister laughed and said, “You have no idea how the law and order, legal
and political systems work in India. I don't want to bore you with the details,
but it's sufficient for you to know that powerful people like me control
everything. You think, by doing what you did, you could strip me of my power?
You think all that drama you created will make people vote against me? You think
the police will arrest me and the court will hang me? Those things happen only
in the movies. Reality is different. I am sure you'll spend many months in
jail—that will give you enough time to ponder over what I said and realize the
futility of messing with powerful people.”

Aryan said, “I don't know when we'll come out, but you can be sure that your
death is in our hands, whenever that may be.”

The minister laughed once again and stormed out of the room. Aryan and Divya
were alone in the room once again. They looked at each other, but still couldn't
speak. Words simply failed them.

While Guru, the minister, was walking out of the prison, he heard someone
calling his name. Someone had the temerity to call him by name instead of
addressing him as 'Sir'? He turned around to see who it was.

“Ah, Mr. Reddy. Thank you so much for your help. These kids had been a nuisance, but
now, because of your help, they are safely locked up in the prison—that's where
lawbreakers belong. I appreciate your swift and proactive actions”, the minister
said.

“Mr. Guru, not all lawbreakers are inside prison. There are people who have
committed larger crimes and are still roaming outside freely”, the police
inspector said.

This inspector not only called him by name but also had the guts to mock him!
Guru maintained a smiling face and replied, “That shows the inefficiency of the
police department.”

The inspector, realizing he couldn't intimidate the minister, lowered his voice
and came straight to the point—“Mr. Guru, I hope you will remember the incentive
you promised me earlier on the phone.”

“Mr. Reddy, you did your job. The government pays you a handsome salary for that. I
pay incentives only to people who do special favours for me, apart from their
regular duties”, the minister said. Flashing a fake smile at the inspector, he
got into his car and left.  The inspector was left fuming inside.