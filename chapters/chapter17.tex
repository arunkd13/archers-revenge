\chapter{}

\lettrine{A}{s they} had hoped, the security guard left in less than five minutes. Aryan and
Divya waited for five more minutes near their bike and walked back swiftly to
the spot. It was almost dark but not totally dark. Even though they couldn't see
the terrace of the Minister's guest house, they knew the party had already
begun—they could hear the music and see the glow caused by the reflection of
lights.


Divya switched on the torch-light in her cell phone while Aryan quickly
assembled both the recurve bows in less than five minutes. They took ten arrows
each and placed it on the ground, next to each other. They took one arrow,
placed it in its position on the bows and lifted their left hand (that was
holding the bow) to an angle of about sixty degrees. Both of them pulled the
string of the bow (along with the arrow) to the requisite length, as practised
earlier. They were finally in position and ready to launch the arrows.

They looked at each other. “Let's start at the count of three,” Aryan said,
“Don't stop until all the arrows are launched. Once all the arrows have been
fired, run.”

Divya nodded her head.

“1… 2… 3.”

They launched the first arrow into the air. It went up in the air quickly,
continued on the upward journey for some time, then moved parallel to the ground,
and began the downward descent towards the terrace of the Minister's guest
house. Even though their arrows were not sharp, they were not totally blunt
either. Since the party was in an open terrace, there was practically no
obstruction to stop the arrows. Even though they couldn't see the arrows landing
on the roof of the guest house, they were sure it would hit someone or something
and cause enough commotion.

They repeated the same procedure and launched all the ten arrows in quick
succession. Once all the arrows were launched, they started running towards the
bike. On reaching the bike, they quickly disassembled the re-curve bows and
placed them inside their bag. Aryan sat in the front of the bike and started it
while Divya sat behind. In less than five minutes after the attack, they were
already on the East Coast Road and were moving towards Mahabalipuram. They
wouldn't know the impact of their attack until the next day morning when it
would (hopefully) be reported in the newspapers.

They sped on the East Coast Road until they reached the Kovalam Junction. Aryan
turned right in order to go towards the Chennai Bypass road that would enable
them to go to the neighbouring state, Andhra Pradesh (where Tirupati town is
located). But once he turned he was stopped by the Police—Traffic Police.

“Show me your licence and documents,” the Traffic Constable demanded.

Aryan took out his licence, copy of vehicle registration book, and insurance
papers.

“The name on the licence and the name on the registration copy don't match. Why
is this vehicle not registered in your name? Whose vehicle is this?” the Traffic
Constable questioned.

“It's mine. I just bought it yesterday. I am yet to transfer the documents to my
name,” Aryan answered.

“How do I know if this bike has not been stolen from somewhere?” the traffic
constable asked.

“It's not a stolen vehicle. If you want I'll call the person from whom I bought
the bike. You can check with him,” Aryan said.

The cell phone of the Traffic Constable started ringing.

“Yes… What?… The Minister's house has been attacked?… The suspect is on
the run?… But how does that concern our department?… OK. If you know the
type of vehicle and the vehicle registration number, I can stop them… It's
impossible to check each and every vehicle moving through this road… No,
without authorization from the higher-ups in my department, I can't do that…”

“I am already working over-time here, and they want me to do the job of another
department as well!” the Traffic Constable mumbled towards no one in particular
after he cut the call.

Just then he noticed the large black bag in Divya's hand. Aryan saw him noticing
it and wanted to divert his attention somehow.

“Sir, I have only Five Hundred rupees,” Aryan offered.

The mention of money, that too so quickly, brought out a smile from the Traffic
Constable. He turned to Aryan and forgot about the bag.

“OK. That will be enough for now. But make sure you transfer the ownership of
the bike to your name immediately. Is that clear?” the Traffic Constable said.

“Yes, Sir.” Aryan said and offered a Five Hundred rupee note to the Constable.
The traffic constable went over to the next 'customer', Aryan started the bike,
and they proceeded towards Tirupati.

“Are we going to ride all the way to Tirupati on this bike?” Divya asked.

“Yes. Bus stand and railway station maybe monitored. Going by bike is safer,”
Aryan replied.

“But they may not know who attacked them, yet,” Divya said.

“No. I think they would have figured that by now,” Aryan said. A drop of sweat
rolled down from Divya's forehead. Did the police already know who they were,
she wondered.