\chapter{}

\lettrine{O}{n} reaching Tirupati, Guru rushed to the hospital he was familiar
with. Since the injuries were not serious, they treated him as an outpatient. He
immediately contacted his sources to gather information about Samara Simha Reddy, or SS
Reddy, as he was more commonly known—the police inspector in charge of Tirupati
town. He called the inspector on his cell phone. After the customary
introductions, the minister came straight to the point—

“I was attacked on the hills—someone tried to kill me”, the minister spoke in
a loud voice.

“Please relax. I am sure you are safe now. Can you give me some details on what
happened?” the inspector asked.

“They punctured the car tires. Once I got down from the car to check if my
people had fixed the tires, I was attacked.”

“Attacked with what? Guns? Knives?”

“No. Bows and arrows. They were archers.”

“Archers?”

“Yes, archers. They were attacking from the mountainous patch, just above the
road.”

“How many of them were there? Did you see any of them?”

“I don't know—I think there were at least two archers. It was difficult for us
to see them from our position on the road.”

“Any injuries?”

“Yes. One of my security guards and I were struck with arrows, but the injury is
not serious. I am at the hospital now. The doctors said we can leave within a
few hours.”

“When and where did this happen?”

“We were attacked an hour ago while coming down from the Tirumala hills. I
think they stopped us around the third or fourth U-turn from the top. I will ask
one of my security guards to stay back and help you identify the exact location
and other details you may need.”

“That would be helpful. Any casualties on their side?”

“We started firing in the direction from which the first arrow was launched. I
think we shot one person—we heard a loud cry. But another arrow hit us from
behind, and that's when we decided to leave the place.”

“Do you suspect anyone?”

“At this moment—no. But it could be related to certain events that happened a
couple of months ago. I'll investigate from my side and will try to give you
some information on the potential suspects. I will be able to do that only after
I reach New Delhi.”

“Can you come to the police station and file a formal complaint?”

“I prefer that you deal with this case privately. There should be no records
whatsoever. I don't want this news to leak out—especially to the media.”

“Why should I do this for you?”

“If you help me with this case, I'll make sure you'll get the promotion that
you've been wanting for the last five years. I'll also make sure you
get a good incentive for this help.”

That promise was sufficient for the inspector to give his full attention to this
case.

Inspector Samara Simha Reddy, loved the authority his job gave him. But until he
became the inspector, he had been made to go through hell—at least in his opinion.

Brash and independent, he hated obeying orders from seniors. He wouldn't obey an
order that he didn't want to, and he even shouted back at his superiors.  His
superiors couldn't fire him—that was a privilege that government employees
enjoyed in India. So, he was transferred to remote corners of the state.  During
those days, he hated his job.

However, after becoming the inspector and having a small region directly under 
his control, he began enjoying his job—perhaps because of the 
independence it offered him. People had to obey his orders and there was minimal interference from his
superiors. Since his attitude was well known, none of his superiors communicated
with him unless they had to. His bitterness towards life and job decreased
considerably. But his attitude ensured he was never recommended for any pay rise
or promotions. He had to be content with what he had. He occasionally requested
his higher authorities for promotions, but as he had anticipated, they routinely
rejected his requests.

Promotion was not his aim. By requesting for a promotion, he hoped they would at
least give him the pay hike he was entitled to—but there wasn't much luck.
Now, with the minister's phone call, he sensed a good opportunity. With the kind
of connections important ministers had with the top officers of the police
department, he should be able to get the pay hike. He might even get the
promotion the minister promised. Not to forget—the incentive. All this for
just doing his job—life was getting better!