\chapter{}

\lettrine{A}{ day} before their scheduled attack, Aryan and Divya travelled from Tirupati to
Chennai in the unreserved compartment of the train. They started early in the
morning and reached Chennai before noon. They had already booked a hotel near
the railway station so that they could check into the hotel immediately after
they reached Chennai. Aryan picked up a second-hand motorcycle that he found in
an online classifieds website, after paying the requisite amount that was
already agreed upon with its owner via email. They wanted to use the bike to
reach the guest house and quickly get away from there without anyone's
knowledge. Since the bike was not registered in their name, they figured it will
be difficult to trace its ownership even if someone managed to note down the
number while trying to escape.

The East Coast Road (ECR) is the entertainment highway of the Southern
Metropolitan city of India - Chennai. The road stretching all the way from
Chennai to Pondicherry (former French colony) is two hundred kilometres long. It
is dotted with various entertainment avenues like theme parks, resorts, beaches
and guest houses. The road has been built along the beaches of the Bay of Bengal
in the eastern coast of India and hence the name. Flanked by the beach on one
side and IT highway (office complexes) on the other, the area around ECR has
residential villas and VIP guest houses. Guru, the Minister, owned a large guest
house on one of the smaller roads that connected the beach and the ECR highway.

Aryan and Divya reached the actual scene of attack and surveyed both the
Minister's guest house and the area on the beach from where they would launch
their arrows. Google Maps had reproduced the area fairly accurately and hence
they were not in for much of a surprise. They parked their bike on the lane
adjacent to the beach and played in the beach water for sometime while
simultaneously surveying the area. As they expected there was no one nearby the
beach. That reassured them that the place would be equally secluded in the late
evening on the next day when they planned to carry out their attack.

There were two more parallel roads connecting ECR highway to the beach lane,
around hundred meters to the left and right of the Minister's guest house. They
decided to use one of them - the left one, to park their vehicle and move out of
the beach after their attack.

The next day, as planned, they came to the same spot around 6:30 in the evening.
The sun had almost set, but there was enough light for them to position
themselves at the right location on the beach from where they planned to launch
their arrows. They carried two re-curve bows in a large black bag (both of them
kept dismantled inside - to be assembled on the spot). The bag kept the bows
hidden from public view. They also brought some heavy aluminium arrows whose
edges they deliberately blunted in order not to cause much harm to the people
partying on the terrace of the Minister's guest house.

As they had expected, the road that led to the Minister's guest house (from ECR
highway) was guarded heavily, and the Police were checking people and vehicles
arbitrarily. They crossed that road and took the next left turn to reach the
beach. There was no Police or security on this road. Aryan parked the bike close
to the beach, and they walked back hundred meters on the beach lane to reach
their predetermined spot. They sat down on the spot and pretended to talk about
silly things in order to appear like lovers to anyone who may pass by. They had
to wait until 6:55 PM to assemble their bows – by that time it should be
sufficiently dark and hence would be difficult for anyone to notice their
activity.

At 6:40 PM, while they were chatting about some inconsequential things, they saw
a guard approaching them.

“Someone is approaching us. What do we do now?” Divya asked.

“I don't think there is any reason for anyone to suspect us yet. Just keep
talking silly things and pretend as if we are lovers casually passing time in
the beach,” Aryan said.

In spite of this decision, they were silent, and their hearts were beating fast.
What if the guard discovers the bows and arrows? Would they be arrested?

As the guard came closer, they noticed that he was not a police. He was probably
a private security guard working with one of the many villas around the place.

“What are you two doing here?” the guard asked.

Aryan replied, “We are friends. We just came to pass some time on the beach.”

“You two are college students?” he asked.

“Yes, same college and same class,” Divya replied.

“Show me your college ID cards,” the guard demanded.

“We didn't come directly from our college. We went home and then came here. Our
ID cards are at home,” Divya said.

By this time, the guard noticed the unnaturally long black bag lying next to
them.

“What's in that bag?” he asked.

“That's my cricket bat and stumps kit. I need to hand it over to a friend on way
back – that's why I brought it,” Aryan said.

“Open the bag, let me see what's inside,” the guard said.

Aryan paused and took a deep breath. There was no way he could show the contents
of that bag to the guard. They were not only carrying bows and arrows, but they
also lied to him.

“Why should I show that to you? Do you have a search warrant? Are you a cop?”
Aryan asked.

The security guard was taken aback for a moment. He then said, “See, this is an
empty beach, and it is not safe for youngsters like you to be roaming around at
this time. It will be better if you two go back to your home now,” the guard
said.

“But, we came just now,” Divya protested.

“It's OK. We'll go back. Thanks for your concern,” Aryan said, hung the bag to
his shoulders, pulled Divya's hand, and started walking towards their bike.

“What do we do now?” Divya whispered once they reached their bike.

“We couldn't stay there – it would have only increased his suspicion. We'll wait
here for ten minutes and then go back. By that time I think he should have
left,” Aryan replied.