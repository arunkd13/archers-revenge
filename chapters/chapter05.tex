\chapter{}

Over the next few days, Aryan spoke to his father's close friends and
colleagues. He told them about the anonymous phone call. While they agreed that
there were problems between the Renewable Energy Department and the
Petroleum/Conventional Energy Department, they couldn't confirm the murder. But,
they informed that there was something abnormal about his father's death. A
couple of them even admitted that they had come across rumors similar to what
was mentioned on the phone to Aryan.

Aryan went to the local police station and told the Police Inspector about the
anonymous call he received and about his suspicion that his father may have been
killed.

“So, you want me to take action against a Minister based on an anonymous call?”
the Inspector asked Aryan.

“No. I just want you to register a case and investigate further,” Aryan replied.

The Inspector looked firmly and said, “See . . . your name is Aryan right? OK.
Think about this – why should that person, whoever called you anonymously, want
to do that? If he was telling you the truth, he could have revealed his
identity. Why didn't he do that? Think about it.”

“He was afraid that he might get into trouble,” Aryan said, “Besides, I
discussed with some of my father's friends and colleagues and they too admit to
hearing rumors that my father was killed.”

“Ah, very good. Now you want me to take action based on rumors?” the Inspector
asked.

“But -”

“You need to understand how the system works Aryan,” the Inspector interrupted.
“Without any credible witnesses, I will not be able to re-open this case. The
report from the hospital is sufficient to prove that your father died due to
heart attack. I can't do anything based on anonymous phone calls or rumors. If
you want me to take legal action, I need witness. I need evidence. Credible
ones.”

Aryan thought it was fair enough. While he was leaving the police station, the
Inspector asked, “So, what are you doing? Are you studying?”

“Yes.”

“Why don't you just continue with your studies? Whatever happened has happened.
If I were you, I would not get into unnecessary trouble,” the Inspector said.

Aryan nodded his head and walked out of the Police Station.

Aryan was still unable to decide if his father was murdered. There was nothing
he could do until he knew that for sure. He hoped the Police might help him find
out the truth, gather evidence, and take action against the Minister (if he was
involved), but the meeting with the Inspector was in vain.

Aryan walked into his apartment when four thugs pulled him towards the compound
wall and surrounded him.

“Who are you people? Why are you dragging me like this?” he shouted.

“Ah, our young detective is angry. Look at his cheeks - it's so red. Guys, be
careful - our superhero might take us all down with his little finger like how
they show in the movies,” One of them said. Others laughed. “Why do you want to
waste your energy? You think you are Sherlock Holmes? It seems you even went to
the Police Station to register a complaint today? People die when they have to -
you can't do anything about it now. Do you want to meet with the same fate?”

Aryan tried to free his hands from the strong grip of two thugs who were
standing on either of his sides. “It's none of your business,” he said.

A man standing in front of him took a knife out of his pocket and pressed the
blunt side against Aryan's throat until he winced with pain.

The thug said, “Yes, it's none of our business. We were paid only to kill you,
not to talk. I guess we should just finish our business without wasting anymore
time. Do you know what will happen if I turn this knife around?”

Aryan looked down at the sharp edge of the knife that was sparkling due to the
reflection of sunlight. Aryan trembled with tears.

Sensing that their mission of threatening him was accomplished, one of the thugs
said, “If you continue your detective work, this knife will pierce through your
throat with such speed that until you wake up in heaven, you'll not know what
happened. Is that clear?”

They meant what they said – Aryan could see it in their eyes. But this incident
confirmed what Aryan feared – his father was indeed murdered by the Minister.