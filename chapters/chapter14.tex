\chapter{}

\lettrine{D}{ivya} found Aryan's makeshift home at the forest interesting. She thought it was
insightful of him to have created a remote base where he could stay safe and
away from the Police, Politician, and everyone else. It will indeed be difficult
to trace him up here. Aryan had allowed the possibility of failure and was in a
position to retreat, prepare, and launch further attacks whenever he was ready.
He was quite resourceful and planned things meticulously. She liked his
methodical and disciplined approach to what was essentially a crime committed on
impulse.

She was also surprised to see how he was able to live in a remote forest for
days together without depending much on the outside civilization. Aryan
occasionally required rice, grains and some vegetables from outside, but
otherwise, he had almost everything else that were required to live. He got
water from the nearby stream, he got some vegetables from his home-garden, there
were abundant fruits in the forest, he got fire-wood to heat water and fry items
while cooking, he had solar cooker to boil rice, pulses and vegetables, he even
had solar panels to power his cell phone, laptop and LED lights. With sufficient
stocks, one might be able to live in this place for months together . . .

“So, do you find this place comfortable?” Aryan asked. They were too tired to
talk on the previous night when they reached Aryan's home in the woods, and had
slept immediately upon arrival.

She looked at him. “It's not as comfortable as a hotel room, but it's more
comfortable than a prison,” she replied.

The very thought of prison, where he might eventually end up for the crime he's
about to commit irked Aryan. But he had already factored in all consequences and
had taken the best possible decision after thorough consideration of
eventualities. He was ready for whatever future might bring to him - it was too
late to change his mind now.

Aryan asked, “You live in Chennai?”

“Yes. My house is in Chennai – I live there with my brothers and my mother,” she
said. “How about you?”

“I was studying in the UK when the news of my father's death reached me. I was
born and brought up in New Delhi – I lived there until about a month ago. But I
realized it was too dangerous to live there, considering what I am doing now.
So, I created this hide-out,” he said.

“It's very thoughtful of you to create this house – I did not think about . . .
a failed mission, you know,” she said.

“Do you still want to kill the minister?” Aryan asked.

“Of course,” she said. “That's the only goal of my life.”

“If you like this place and if both of us have the same goal, why don't you stay
here? You'll be caught if you go back to Chennai. It will be easy for them to
associate us with the attempted murder - they will come after us,” he said.

“Yes, you told me that yesterday. I think you are right – staying in the city
henceforth will be dangerous,” she said.

“It's not only about that – we can plan and attack him together. Our combined
forces will be more effective than us working independently,” he said.

She thought about it for a moment and said, “I prefer to work alone. I don't
like things being thrust upon me.”

He said, “Why do you think I will thrust anything upon you? We can brainstorm
ideas, discuss strategies, and execute only those that both of us are convinced
with.”

She looked away from him and considered the offer for sometime. She still
appeared doubtful and was unable to come to a firm decision. He wondered if she
could have some other concerns.

He said, “Are you afraid to live alone with a young unmarried guy?”

She was surprised to hear that question. She smiled, looked at him and said,
“Are you afraid to live alone with a young unmarried girl?”

He was equally surprised. “I am not.”

She said, “Why should I be?”

He was reminded of the brief assassination attempt on his life. She almost
pierced the knife into his neck. No wonder she was not afraid!

“So, would you want to stay?” he asked.

She thought for a while and then said, “Considering the fact that I may be
caught if I go back, I think I will rather stay here until our business is
finished. But, I don't like being compelled to do anything and I should be able
to leave anytime I want. Is that fine with you?”

He offered his hand to her and said, “Welcome to Revengeville. As a special
offer, I have decided not to charge any rent – you can stay here for free as
long as you want.”

She shook his hand and mocked back - “You cannot charge any rent even if you
want to. You are not the owner of this place anyway.”

He was glad that she agreed to stay on. For the formidable task that lay ahead
of them, they could very well help each other. Two people can execute attacks
more effectively than a single person. Besides, loneliness was getting on his
nerves and he needed some company to stay sane.