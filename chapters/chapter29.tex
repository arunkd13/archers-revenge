\chapter{}

\lettrine{O}{nce} they descended the mountain and reached Tirupati, Aryan rode the bike to
the airport and parked the vehicle. Divya finally asked him the question that
was lingering on her mind all this while.

“Why didn't you kill him?”

Aryan got down, looked into her eyes, and looked away. He said, “I couldn't. I
wanted to kill him right until the last moment, but when I looked at you, I
realized I was… maybe… wrong.”

After a brief pause, he continued, “I want to live. I want a future. A future
that involves both of us. Let's go to UK. My study visa hasn't expired yet—you
can take a travel visa for now. I am sure we'll be safe there.”

Divya asked, “Why do you want me to come with you?”

Aryan said, “Don't you realize what we just did? Or for the matter, what we have
been doing all this while? If we stay in India, we'll be caught sooner or later
and we'll be put inside the prison for God-knows-how-many-years. We attempted a
murder, Divya”.

“Don't worry about me. I'll be safe. You go back to UK and stay away from India
for a while,” Divya said.

Aryan said, “It's not just about our safety, Divya. It is… I am in lo-”

Divya said, “Don't say that Aryan. It's… not possible.”

His eyes widened in disbelief. He wanted to confirm her decision once again,
“Are you sure?” he asked.

Divya looked away. “Yes. I am very sure,” she said.

Aryan's shoulders dropped. He looked down and was breathing deeply.

After a few seconds of silence between them, Aryan turned back and mustered the
courage to ask, “Why Divya? You… you… don't like me?”

Divya looked back at him and said, “It's not that Aryan. It's just that I…
I have promised my dad that I'll keep my personal and professional lives
separate.”

“What professional life are you talking about? I thought you said your father
was killed along with my father.” Aryan said.

“No… he wasn't. He's still alive,” Divya said and looked down. Aryan's face
moved forward, and his eyes focused on hers.

She continued, “I am not the daughter of the Energy minister. He has no
daughters. He only has two sons and both of them are settled abroad,” she said.

“What are you saying Divya? Who are you then? And what were you trying to—were
you set up by someone to kill me?” Aryan asked.

“No. Actually, I was set up to save you. My father works for the CBI—Central
Bureau of Investigation. I finished my graduation sometime back and plan to
appear for the Civil Services examination. I want to take up a career in the 
IPS—Indian Police Service. In the meantime, I am… sort of… interning with
the CBI. I am helping them with this case,” she replied.

It took a while for Aryan to even consider the possibility of her working for
the CBI. He recollected the events that happened over the last few months
involving both of them. He still couldn't believe that she could've been set up
by the CBI to track him. If the CBI knew where he was and what he was up to,
they should have arrested him by now. Not send a person to help attack a
minister!

“I don't get it,” Aryan said, “Initially, I found you on the mountainous patch
in between the U-turn when you were trying to attack the minister like I was.
How come you claim to work for the CBI then?”

Divya replied, “Don't you think my releasing of the arrow on the minister, just
before yours, at the same time and location, was too much of a coincidence?”

Aryan said, “Yes, but no one knew I was going to attack the minister on that day
from that place. How come… how come you claim it was not a coincidence? I
mean, how did you come to know about it?”

Divya said, “Aryan, one minister and two senior executives of the Renewable
Energy Department were killed, and you thought the Government will not do
anything about it? The CBI was instructed to investigate on this case by the
prime minister. We were able to gather all info on who was responsible for the
killing and why they did it, but we didn't have any evidence to prove that the
minister, Guru, was behind these killings. Our only evidence, the two employees
who arranged the tea to be poisoned, were also killed shortly afterwards. Since
the minister had a considerable clout and influence we couldn't take any legal
action against him without solid evidence.”

Aryan said, “OK. But what does that have to do with tracking me?”

Divya continued, “Our contact in the police station in Delhi informed us about
you and your complaint that was never filed on the day you went to the police
station. That's when we started tracking your mobile calls and online
activities, mainly to protect you, but also to track you and try and get some
evidence against the minister. Your browsing activity on the Internet gave us
sufficient cause for concern. We figured soon enough that you were planning to
kill the minister.”

“OK. But how did you find out that I was going to attack him on the Tirupati
hills?” Aryan asked.

Divya said, “Your mobile GPS. We were tracking your location using the mobile
GPS which you didn't bother to switch off, for most of the time. We knew when
you shifted to Tirupati and in which hotel you stayed. But we couldn't locate
exactly where you stayed in the hills. We kept monitoring your activity. When
you came out to survey the U-turn on the road, that too, three days before the
minister was scheduled to visit the Tirupati Temple, we had a good reason to
suspect that you may attack him.”

Aryan said, “But still—how did you find out the exact position from where I
would launch my arrows? I think you were very close—just a few meters ahead of
my location.”

“We installed a few wireless day and night IP Surveillance cameras on the trees
around the U-turn. When you rehearsed your attack on the two nights before the
actual date, we knew exactly how you planned to attack him and from where. On
the day of the attack, I deliberately positioned myself ahead of your position,
without your knowledge of course, and released the arrow on the minister's
shoulder before yours could hit his heart. My attack was a ploy to distract you
from killing. We didn't want an innocent person like you spending all his life
in jail. Besides, that gave me an opportunity to join you and monitor all your
activities closely,” Divya said.

“Why did you bother doing all that? If you knew I was going to attack the
minister you could have arrested me. Why didn't the CBI do that?” Aryan asked.

“Arresting you was not our priority. We just wanted to make sure that you didn't
murder him. Besides, we wanted to gather some evidence against the minister and
figured that working with you might be useful,” Divya said.

“What about those two attacks that we launched together later on? We attacked in
his guest house and in the election rally. Why did you coordinate with me for
those two?” Aryan asked.

“Did we even attack him in his guest house? We just launched some arrows, that
too with blunt arrowheads, all around him. The first attack was supposed to keep
you engaged and prevent you from suspecting me. But his panic and reactions made
us realize that we could use the media against him. That allowed us to plan a
more detailed and purposeful second attack which we carried out during his
election rally. That was successful—it weakened him and made him lose the
source of all his power—his position as the minister. Fortunately, we were
able to create a negative opinion about him and people voted against him. That
considerably strengthened our stand against him.”

Aryan asked, “But you still don't have any evidence against him, do you?”

Divya replied, “Do you remember the conversation Guru had with us in the prison
just before the election results were announced? If you recollect that
conversation you'll realize that he admitted to his crime of killing your
father.

“So?” Aryan asked.

“I recorded the entire conversation on my mobile phone,” Divya replied.

Aryan said, “How did you get your cell phone inside the prison?”

Divya said, “Common Aryan. I work for the CBI. Smuggling a cell phone inside the
prison is not very difficult. Actually, I was advised to stay inside the prison
anticipating something like this might happen. I was regularly in touch with my
superiors, my father actually, using the cell phone. We decided not to involve
the police Department in this case. If Nadeem had not taken us out on bail, the
CBI would have taken me out via the court—legally.”

Aryan said, “What about today's attack? I almost killed Guru. Probably I would
have done that if I had not turned back to look at you in the last moment.”

Divya said, “That was the only chance I took personally. I tried convincing you
not to kill, but you wouldn't listen. I was asked to tie your hands and legs
while you were asleep and detain you within your house in the hills so that you
couldn't have done anything today. But I refused to do that. Since I had a fair
idea about you, I took a chance. But I'll have to admit that until the last
minute I had no idea what you'll do. Fortunately, you decided against killing
him.”

Aryan said, “You took a big risk. Until the last moment when I actually changed
my mind, I very much wanted to kill him.”

Divya said, “There is another reason why I took that risk. The minister should
have been arrested as soon as he walked out of the Tirupati Temple—we already
had the evidence required to prosecute him. But the arrest warrant was delayed.
By the time we got the warrant, he had already begun travelling downhill. I was
counting on him getting arrested so that this attack wouldn't be necessary. But
that delay almost cost Guru, his life; and for you, your future. By the way,
Guru has been arrested from the very spot where we left him.”

Aryan said, “So, am I also going to be arrested before leaving the country?”

Divya said, “No. In spite of your misadventures, it was primarily because of you
we were able to gather evidence and arrest the minister now. Had you killed him
there, we would have been no other option, but to arrest you. But since you
chose not
to do that, you'll be safe as long as you are outside India. We'll make people
believe that you escaped from here before we were able to get to you. Go Aryan.
Go back to UK. Live your life just as your father wanted you to. Make your
father proud.”

Aryan looked into her eyes for a couple of seconds. He lowered his eyes, turned
away and walked towards the ticket counter. Not once did he turn back to look at
Divya after that. But Divya couldn't stop looking at him until he left her view.
She was looking in his direction even after he left. How much she wished she
could have accepted his proposal and moved to UK along with him. But she knew
she couldn't do it—there was a career, her chosen career, ahead of her, and she
didn't want to spoil it. And of course, she had a promise to keep—she would
keep her personal life separate from her professional life.