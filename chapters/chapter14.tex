\chapter{}

\lettrine{E}{ven} though, Guru, the minister, had initiated investigations through the
police, he rightly guessed that his personal investigations might be more useful
in identifying the suspects in this case. Only he knew his enemies, and more
importantly his crimes.

Initially, it was difficult for him to guess who might have been behind the
assassination attempt. Not that he had no enemies in his profession, everyone
were friends one day and enemies on the next. People frequently oscillated
between friendship and enmity, based on their commercial interests and Guru's
position. Hence, any of his 'associates' or 'well-wishers' may have been behind
the assassination attempt. Except in rare cases, even in his profession
(politics), people didn't kill others due to commercial losses or losses due to
power politics. Besides, it was difficult to get away with a murder. Hence, he
ruled out professional and commercial motivations.

He thought about all the recent crimes he had committed. All of them were
related to money, except one before a few months when he killed three people in
order to take control over the Renewable Energy Department. A personal loss (of
a loved one) could be a powerful motivation for revenge.

He asked his people to conduct some preliminary investigations on the immediate
family members of the people whom he had killed. At least in one
case—Aryan's—there was sufficient cause for suspicion. Aryan had tried to register
a police
complaint and the minister sent his men to make him silent. One of his men
contacted the University where Aryan was studying and was informed that Aryan
had applied for a long leave on health grounds. His apartment at New Delhi had
been locked over the last couple of months and the neighbours hadn't seen him
around during that time. No one knew where he was.

The minister wondered if he should have killed Aryan. But there were no signs of
any violence from that kid back then—he made sure Aryan was sufficiently
threatened. The minister also discovered Aryan's background in archery training
and was able to make the association with the way he was attacked.

Inspector Samara Simha Reddy, had already arranged a thorough check on the
spot where murder was attempted. Except the nail cluster which was used to
puncture the tires of the car and a couple of arrows, they could not recover
anything else. There were no fingerprints on either of the objects—whoever
attacked was probably wearing gloves. There was no body, no blood, no other clue
that could lead to the suspects.

Just as the inspector was pondering upon how to proceed with the case, the
minister called him and informed him about the probable suspect. But the
minister didn't tell him about the context of the crime or why Aryan was trying
to kill him. It didn't take much time for the inspector to find out. In India,
it may be possible to escape from being convicted of a crime (if one had enough
money and influence) or delay the judicial process, but it's impossible to stop
the truth from circulating around. The inspector only had to talk to a couple of
his friends and their references in Delhi and do a little research to establish
the circumstances and motivations behind the crime. What happened now was an
attempt to revenge the minister for the murders committed by him a few months
back. Now that the inspector knew the truth, he would use it to demand more
payment from the minister eventually.