\chapter{}

\lettrine{A}{s promised} by Nadeem, Aryan and Divya were released on the next
day, and they
went straight to their makeshift house in the hills. Though the police had
searched the premises, they didn't take anything away from there. Not even their
bows and arrows. Almost everything was intact and that was sufficient to
continue living there for the time being.

“So, what are we going to do?” Divya asked and looked at Aryan.

“What more could we ask for? Yesterday morning we didn't know how many months or
years we'll be behind the bars, and today we are already back to our base. We
should utilize this golden opportunity. We will strike one last time—this time
we should kill Guru. That's also what the new minister, Nadeem, wants us to do,”
Aryan replied.

Something about Divya's look told Aryan that she was not comfortable with the
plan. “Do you trust him—the new minister, Nadeem?” she asked.

“Of course, I don't. How can we ever trust a politician? But we should also take
our current situation into consideration. Nadeem wants the old minister to be
eliminated, so do we. He has already helped us by taking us out on bail, and we
will have to help him back,” Aryan said. He wondered why he had to defend his
intentions with her. Didn't she also want the same thing?

“Don't you realize what's happening here? We have now become pawns in their
game. The new minister is using us to eliminate his rival. Even if we complete
the job—kill Guru—and eventually get caught by the police, it's us who will
be prosecuted. Nadeem will not even be questioned. There is no evidence to
connect him with this case,” Divya said.

“Yes, but from when did we start worrying about the consequences of our crime?
If we had killed him the first time, when your arrow missed his heart and
wounded his arm instead, I guess the consequences would have been pretty much
the same. We didn't bother about it then—why bother now?” Aryan asked.

“The situation then was different, Aryan. But now, we have exposed Guru in front
of the national media, and we made him lose the elections—people have voted
against him unanimously. He lost all his power—his dearest ministerial berth
is no longer his. We have already inflicted sufficient punishment on him. Don't
you think that's enough?” Divya asked.

“How can that be sufficient? He killed our fathers, and he will have to pay for
it with his life—nothing less. By killing him we'll convey a powerful message.
Let everyone understand what will happen if they start taking lives just because
they think they are powerful,” Aryan said.

“Did you ever consider what will happen to us after we kill him? What kind of
future do you anticipate for either of us? We will be behind bars for many, many
years. We may be even hanged. Nadeem or anyone else won't come to bail us out
then,” Divya said.

“I don't know why you are suddenly concerned about the consequences of our
crime. You seem to have changed your mind. For some reason, you don't want to
kill him now,” Aryan said and paused to think for a couple of seconds. “Or
maybe, you never wanted to kill him. Guru is alive today because of you. During
the first attack, if your arrow had not hit his shoulders, mine would have
pierced his heart. When we attacked him during the election rally, if we had not
used the rubber pieces on the edge of our arrowheads, Guru would have been dead
by now.”

Divya replied, “The first time, if he had not turned suddenly, my arrow would
have hit his neck instead of his shoulders. It was sheer bad luck. In our recent
attack, although we managed to land both the arrows right on him, we were not
sure about the accuracy from that distance. If he had faster reflexes, someone
had pushed him away, or if we had miscalculated our aim, Guru could have easily
escaped. We had discussed this at length before our attack—we concluded
that our chances of killing him from that distance was very less. Going for his
life and missing it would achieve nothing. It's because we didn't kill him that
we were able to turn the public opinion against him, and he lost the elections.
We had analysed all these things before. You never had a problem then.”

Aryan was silent.

Divya continued, “We have lost nothing yet, Aryan. If we disappear from the
scene, police will start investigating Guru's murders. That will be Nadeem's
next move—he doesn't have any other option. I think we should stay away from
Chennai, Tirupati, Delhi and live anonymously for some time. I am sure we'll be
forgotten after a few months.”

Aryan said, “You are telling me that police will act against him, and we'll get
justice? You know how long it takes to prosecute a person through the legal
system in India. The legal route, how much ever enticing that may seem to you,
is totally ruled out as far as I am concerned. You still trust the legal systems
in spite of what happened to us?” Aryan paused for a couple of seconds. “I also
remember you telling me then that we will strike and kill him once he is
stripped off his power and that should be our logical next step. We made him
powerless so that we can kill him. Now he has been stripped off his power, but
you seem to have conveniently forgotten what you said.”

Divya lowered her eyelids and looked down.

Aryan continued, “Guru is now no longer a minister and is no longer protected by
the police. Can't you see this is our best chance? If we don't do anything now
most probably he will find this place and kill us both. It won't take a long
time before someone in the police gives away our location. Either we strike or
wait for him to strike. Which option do you think is better?”

Divya said, “Let's go away from here. Let's go to some place where no one can
find us. Let's give justice a chance, Aryan. We need to be more patient—we
can't expect everything to happen overnight.”

Aryan asked, “Why are you suddenly inclined to solve this legally? I don't
understand how you still believe in the police and law. We can never be sure if
the police will take any action against Guru or not. But we can always be sure
with our bows and arrows.”

Divya asked, “If your father were alive today, do you think he would have
approved your actions? Do you think he will be proud of what you are doing?”

Aryan was silent.

“Can you live knowing that you have killed a person? Won't your conscience
trouble you until the day you die?” She continued. “What's the difference
between you and him? Guru chose to murder fellow human beings. In the name of
revenge, you too are doing the same thing now. What's the difference between you
two, Aryan? Is violence the answer for violence?” she asked.

Aryan looked away silently admitting that he couldn't find words to craft a
reply to all the difficult questions posed by Divya. She asked the questions
that Aryan did not have the courage to ask himself.

After a short time, he turned back to look at her and said, “Let me make it
clear. I am going to kill the minister whether you help me or not. You will have
to decide if you want to work along with me. The choice is yours—if you don't
want to be a part of this, you are free to go anywhere right now,” Aryan said.

Divya looked away. After a brief pause, she turned to face Aryan and asked him,
“Have you made any plan for our next attack?”

Aryan was relieved. He smiled at her and said, “The minister will be coming to
Tirupati Temple once again in two weeks for his wedding anniversary. This time,
unlike our first attempt, luck is not going to favour him.”

“We are going to attack him on the same spot?” Divya asked.

“Yes. That's where he will not expect us to attack, once again,” Aryan said and
winked at her.