\chapter{}

\lettrine{T}{wo weeks} later, Aryan and Divya were back in Chennai. They rode
their bike from Tirupati to Chennai. They checked into the same hotel near the
railway station where they had stayed earlier. By evening, they were ready to go
to Mount Road, where Guru, the minister, was scheduled to address a large
election rally.

Mount Road, or Anna Salai, is an important road that runs through the heart of
Chennai. Many businesses have their offices on this road. Mount Road connects
important areas of the city and, hence, is generally busy with a lot of vehicles
plying on it throughout the day. Since it was difficult to get permission
to hold a political rally on the main road, Guru managed to arrange his meeting
on a small by-lane off the main road. Traffic on that lane was blocked to
accommodate the rally.

A large temporary stage with huge banners of the minister on both sides was
constructed a few meters before the far end of the lane, and chairs were laid in
straight rows, one behind the other, in front of the stage. Political cadres,
well-wishers, the public and media who came to listen to the minister talk had
already occupied most of the chairs. Many people were standing on the sides and
behind the chairs—the crowd almost reached the main road. Since this meeting was
at the heart of his constituency, and it was his last rally before the
elections, it was very important for the minister to create a positive
impression. Guru made sure the arrangements were grand and the crowd was huge.

Since Guru was a Cabinet minister, this event was covered by both the local and
national media. TV journalists, newspaper and magazine reporters, bloggers,
social-media writers and other members of the media were present in full
strength to cover the event. Guru personally oversaw all the invitations to
ensure maximum publicity.

When Aryan and Divya reached Mount Road, the meeting had already started.
Introductory speeches were ongoing, and the minister was scheduled to speak shortly.
Right opposite, across Mount Road, overlooking the stage, was an old government
building with three floors. Aryan chose this particular building because its
terrace offered a direct line of sight to the meeting stage, and the building was
empty. The government department which occupied the building earlier had been
relocated, and the building was about to be sold off through an auction. An empty
building was a good choice for attacking the minister without attracting any
unwanted attention.

Aryan and Divya parked their vehicle in the neighbouring mall, two buildings
away. They were wearing black jerseys and black jeans. They carried their
archery gear in a black travel bag. They walked along the main road towards the
empty Government building. Aryan noticed a police constable sitting inside the
Government building complex. The gate of the building was closed but not locked.

There was an ATM in the neighbouring building with a private security guard. The
wall between the two buildings was almost five feet tall. Since some bikes were
parked along the wall near the ATM, Aryan was confident that he could climb on a
bike and jump over to the next building easily. Aryan quickly thought of a plan
and whispered it to Divya.

Divya opened the gate of the empty government building and walked towards the
police constable.

“Who are you? What do you want?” the police constable asked.

“Sir, I am looking for this address,” Divya said, handing a paper to him.

He looked at the paper and read its contents—'Express Avenue Mall,
Royapettah'. He then looked at her. She maintained an innocent, smiling face.

He started explaining the way to reach the mall, while Divya kept asking further
questions, pretending to be new to the city. In the meantime, Aryan went and
stood in front of the ATM in the neighbouring building. The security guard was
busy elsewhere. Aryan quietly slipped into the side lane of the building and
jumped over the wall, crashing into the neighbouring government building compound.
He jumped high and landed on the ground once again to ensure the noise was
loud enough to be audible.

The police constable heard the thud, asked Divya to wait for a minute, and went
to the side of the building. Divya paused for a few seconds and then followed
him. On seeing Aryan, the police constable walked towards him.

“Who are you? Why did you jump over the compound wall?” the constable asked.

“I am a hobbyist jumper. Jumping walls is my passion,” Aryan replied.

“What?!” the police constable said and tried to grab Aryan's shirt collar.

Aryan shoved the police constable's hand and pushed him behind. The constable's
back hit the wall of the building, but he recovered quickly and moved forward
with both his hands stretched out in an attempt to grab Aryan. Aryan moved out
of his way and pushed the constable from behind. This time, the constable fell
down on the ground. Aryan took a chloroform-soaked napkin and tried to cover the
face of the constable with it. But the constable caught Aryan's neck and pushed
him down.  The constable rolled over Aryan, held Aryan's hand tightly, and was
about to punch him, when Divya sprayed pepper spray on the constable's face. The
constable felt the burning all over, loosened his grip on Aryan, and moved his
hands quickly to cover his face. Aryan got up. Both Aryan and Divya joined
forces to push him down. They removed his hands, pressed the chloroform soaked
towel on his mouth and nose until he fainted.

Aryan and Divya tied the constable's hands and legs with a rope and stuffed a
handkerchief into his mouth. They looked around to see if anyone had noticed, but
to their relief no one was in sight. They carried him and left him behind 
the building. It would be a few hours before the constable regained consciousness,
but by that time they would have left the building. Aryan and Divya reached the
stairs and climbed until they reached the terrace of the building. The terrace
was locked. It had an old wooden door with holes all over it—probably the
department didn't think it was necessary to protect the terrace. Aryan removed
an iron hammer from his bag. It took him just two swings at the lock to break
it.

From the terrace, they could see the Mount Road below them. The lane, which was
to be their meeting venue after the attack, was opposite to them. There were
people sitting and standing on the lane, facing the stage. The stage was 50–60
meters away from their location, and they could also see people sitting on the dais,
waiting for their turn to speak. Though they couldn't clearly identify the faces of
people on the stage, they would be able to identify the minister when he
rose to speak because the loudspeakers blared the speeches around the area.
Aryan knew the voice of the minister by heart. They quickly assembled their bows
and placed their arrows on the floor next to them. Now, they were waiting for
Guru to stand up and give his speech on the microphone.

They didn't have to wait for long. Within fifteen minutes, the minister was invited to
speak. All the cameras focused on him. A few news channels were broadcasting the
meeting live. The opportunity that Aryan and Divya were waiting for had finally
arrived. They took one arrow each, placed it on the bowstring, pulled the bow
string to the requisite length, adjusted their aim, and got into striking
position.

“At the count of three,” Aryan said.

Divya nodded in agreement.

“1… 2… 3.”

Aryan and Divya released their arrows at the same time. The arrows moved quickly
through the air, and within just a couple of seconds, one arrow hit the minister's
chest while the other hit him in the stomach. The minister fell to the
stage, his back hitting the floor, unable to bear the force of the impact.

“Bingo,” they said in unison, and quickly ducked behind the parapet wall to hide
from anyone looking in their direction. They then moved five meters
behind, hoping to be less visible to the people and the video cameras. They took
ten arrows each, loaded them on their bows and released all of them in quick
succession. Those twenty arrows were not aimed at anyone in particular but were
released in the direction of the crowd.

After launching all the arrows, they dropped their bows in the same
location, removed their black Jerseys and jeans and came out in the normal
clothing they had worn underneath. They quickly raced downwards, took the rear
exit, jumped into the adjacent compound and ran towards the gate.

As expected, there was commotion all around. People were running out from the
meeting venue in all directions. Aryan and Divya walked to the adjacent mall
where they had parked the vehicle, started their bike, and drove away in the
opposite direction on Mount Road before the police could cordon off the
area. That was one big escape!

But they were not aware that surveillance cameras at multiple locations had
recorded all their movements—both in the mall and the neighbouring compound.